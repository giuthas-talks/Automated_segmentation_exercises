\documentclass[12pt,a4,english,finnish,pdflatex%,handout
]{beamer}
\definecolor{MyGreen}{RGB}{50, 120, 50}
\usecolortheme[named=MyGreen]{structure}

\usepackage{babel}
\usepackage[utf8]{inputenc}
\usepackage[T1]{fontenc}
\usepackage{amsmath,amssymb} 
\usepackage{animate}
\usepackage{multimedia}

\usepackage{natbib}
\bibpunct[: ]{(}{)}{,}{}{}{;}

\usepackage{tikz}

\usepackage{tipa}

\usepackage{hyperref}

\setbeamertemplate{navigation symbols}{}

\graphicspath{{figures/}}

\setlength{\leftmargini}{0pt}
\setlength{\leftmarginii}{1em}

%% Write out the names of graphics files being included 
\newwrite\graphics
\immediate\openout\graphics=\jobname.graphics%
\let\oincludegraphics\includegraphics% store original \includegraphics
\renewcommand{\includegraphics}[2][]{% prepend to it (could also use xpatch, etc.)
  \immediate\write\graphics{#2}
  \oincludegraphics[#1]{#2}}

\newcommand{\kommentti}[1]{
  {\bf[#1]}
}


\title{Automated Segmentation Exercises with PATKIT}
\author{Pertti Palo} 
\date{2,3,5 June 2025} 


\begin{document}

\frame{\titlepage
  \centering
} 

\frame{\frametitle{Introduction: The what and why}
  \begin{itemize}
  \item Segmentation can be explained, but actually getting good at it requires
  hands-on practice.
  \item While Praat \citep{BoersmaWeenink-PraatDoingPhonetics-2010} has an
  excellent segmentation interface, it does not provide a segmentation exercise
  interface.
  \item PATKIT \citep{PaloEtAl-PATKITPhoneticAnalysis-2025} copies Praat's
  segmentation interface and adds a resettable exercise feature. 
  \end{itemize}
}


\frame{\frametitle{Introduction: The how}
  \begin{itemize}
  \item  
  \end{itemize}
}


\frame{
  \centering
  {
    \bf \Large 
    \usebeamercolor[fg]{title}
    Demo - seeing is believing
    
    \vfill
  }
}


\frame{
  \centering
  {
    \bf \Large 
    \usebeamercolor[fg]{title}
    Want to have a go?
    
    \vfill
  }
}


\frame{
  \centering
  {
    \bf \Large 
    \usebeamercolor[fg]{title}
    Thank you!
    
    \vfill
  }
}


\frame{\frametitle{References}
  
\bibliographystyle{apalike}
\bibliography{main}

}

\end{document}

