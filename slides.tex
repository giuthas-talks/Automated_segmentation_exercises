\documentclass[12pt,a4,english,finnish,pdflatex%,handout
]{beamer}
\definecolor{MyGreen}{RGB}{50, 120, 50}
\usecolortheme[named=MyGreen]{structure}

\usepackage{babel}
\usepackage[utf8]{inputenc}
\usepackage[T1]{fontenc}
\usepackage{amsmath,amssymb} 
\usepackage{animate}
\usepackage{multimedia}

\usepackage{natbib}
\bibpunct[: ]{(}{)}{,}{}{}{;}

\usepackage{tikz}

\usepackage{tipa}

\usepackage{hyperref}

\setbeamertemplate{navigation symbols}{}

\graphicspath{{figures/}}

\setlength{\leftmargini}{0pt}
\setlength{\leftmarginii}{1em}

%% Write out the names of graphics files being included 
\newwrite\graphics
\immediate\openout\graphics=\jobname.graphics%
\let\oincludegraphics\includegraphics% store original \includegraphics
\renewcommand{\includegraphics}[2][]{% prepend to it (could also use xpatch, etc.)
  \immediate\write\graphics{#2}
  \oincludegraphics[#1]{#2}}

\newcommand{\kommentti}[1]{
  {\bf[#1]}
}


\title{Automated Segmentation Exercises with PATKIT}
\author{Pertti Palo} 
\date{2nd June 2025} 

\begin{document}

\frame{\titlepage
  \centering
} 


\frame{\frametitle{Land acknowledgement}

  The University of Alberta, its buildings, labs and research stations are
  primarily located on the territory of the Néhiyaw (Cree), Niitsitapi
  (Blackfoot), Métis, Nakoda (Stoney), Dene, Haudenosaunee (Iroquois) and
  Anishinaabe (Ojibway/Saulteaux), lands that are now known as part of Treaties
  6, 7 and 8 and homeland of the Métis. The University of Alberta respects the
  sovereignty, lands, histories, languages, knowledge systems and cultures of all
  First Nations, Métis and Inuit nations. 
  \vfill
  In addition to our university's written land acknowledgement I'd like to speak
  of my own relation to the lands where I have been working for almost a year now.

}


\frame{\frametitle{Outline}
  \begin{itemize}
  \item Land acknowledgement
  \item This slide
  \item Introduction: The what and the why
  \item Method: The how
  \item Demo
  \item Want to have a go yourself?
  \item MaTiPSS ad
  \item Thanks and references
  \end{itemize}
}


\frame{\frametitle{Introduction: The what and the why}
  \begin{itemize}
  \item Segmentation can be explained, but actually getting good at it requires
  hands-on practice.
  \item While Praat \citep{BoersmaWeenink-PraatDoingPhonetics-2010} has an
  excellent segmentation interface, it does not provide a segmentation exercise
  interface.
  \item PATKIT \citep{PaloEtAl-PATKITPhoneticAnalysis-2025} copies Praat's
  segmentation interface and adds a resettable exercise feature. 
  \end{itemize}
}


\frame{\frametitle{Method: The how -- Setting up}

\begin{itemize}
  \item Open a directory with matching TextGrids and audio files in PATKIT as
  an exercise:\\
  ~~Exercises $\rightarrow$ Import directory\ldots
  \item If setting up for others, go to:\\
  ~~Exercises $\rightarrow$ Save as exercise \ldots\\
  to save the audio files and the scrambled TextGrids in a new location. 
\end{itemize}
}


\frame{\frametitle{Method: The how -- Running an exercise}
\begin{itemize}
  \item Either do the setup steps or get an exercise dataset from someone.
  \item If you want to compare your segmentation with the model, you'll
  need to get the model files too.
  \item Do the segmentation.
  \item Check how you are doing by going to:\\
  ~~Exercises $\rightarrow$ Compare to model
  \item When wishing to re-run an exercise go to:\\
  ~~Exercises $\rightarrow$ Scramble TextGrids
  \item Rinse and repeat.
\end{itemize}
}


\frame{
  \centering
  {
    \bf \Large 
    \usebeamercolor[fg]{title}
    Demo - seeing is believing
    \vfill
  }
}


\frame{\frametitle{Discussion}
\begin{itemize}
  \item So, do you have thoughts?
  \item What would make this more useful to you?
  \item What other features would be good to have?
\end{itemize}
}


\frame{
  \centering
  {
    \bf \Large 
    \usebeamercolor[fg]{title}
    Want to have a go yourself?
    \vfill
    --
    \vfill
    Keep an eye out for version 0.18 and updates (0.18.x) later this week.
    \vfill
  }
}


\frame{\frametitle{Installation instructions}
\begin{itemize}
  \item Install uv \url{https://docs.astral.sh/uv/}
  % \begin{itemize}
  %   \item Mac/Linux: 
  %   \begin{verbatim}
  %     curl -LsSf https://astral.sh/uv/install.sh | sh
  %   \end{verbatim} 
  %   \item Windows:
  %   \begin{verbatim}
  %     powershell -ExecutionPolicy ByPass -c "irm https://astral.sh/uv/install.ps1 | iex"
  %   \end{verbatim}
  % \end{itemize} 
  \item On the commandline run:\\
    uv tool install patkit\\
    patkit
  \item This should print the commandline help.
\end{itemize}
}


\frame{\frametitle{Running the example assignment/exercise}
\begin{itemize}
  \item Get the example data from recorded\_data/assignment\_example and
  exercises/minimal and put it in a folder of your own choosing. 
  \item Run:\\
    patkit exercise [folder\_name]
  \item This should open the annotator GUI.
\end{itemize}
}


\frame{
  \centering
  {
    \bf \Large 
    \usebeamercolor[fg]{title}
    MaTiPSS ad
    
    \vfill
  }
}


\frame{\frametitle{Methods and Techniques in Phonetics of Signing and Speech}
  \begin{itemize}
    \item \url{https://matips.org/}
    \item Title and authors 19 July, anywhere on Earth
    \item Full one-page abstract 26 July, anywhere on Earth 
    \item Results by beginning of August.
    \item Conference 17-19 October, University of Alberta, Edmonton, Alberta, Canada.
  \end{itemize}
\vspace{0.5cm}
\begin{itemize}
  \item Canadian Acoustics Week happens to be in neighbouring Calgary 15-17 October.
  \begin{itemize}
    \item \url{https://jcaa.caa-aca.ca/index.php/jcaa/announcement/view/91}
    \item There will probably be a bus between the conferences.
  \end{itemize}
\end{itemize}
}


\frame{
  \centering
  {
    \bf \Large 
    \usebeamercolor[fg]{title}
    Thank you!
    
    \vfill
  }
}


\frame{\frametitle{References}
  
\bibliographystyle{apalike}
\bibliography{main}

}

\end{document}

