\documentclass[12pt,a4,english,finnish,pdflatex%,handout
]{beamer}
\definecolor{MyGreen}{RGB}{50, 120, 50}
\usecolortheme[named=MyGreen]{structure}

\usepackage{babel}
\usepackage[utf8]{inputenc}
\usepackage[T1]{fontenc}
\usepackage{amsmath,amssymb} 
\usepackage{animate}
\usepackage{multimedia}

\usepackage{natbib}
\bibpunct[: ]{(}{)}{,}{}{}{;}

\usepackage{tikz}

\usepackage{tipa}

\usepackage{hyperref}

\setbeamertemplate{navigation symbols}{}

\graphicspath{{figures/}}

\setlength{\leftmargini}{0pt}
\setlength{\leftmarginii}{1em}

%% Write out the names of graphics files being included 
\newwrite\graphics
\immediate\openout\graphics=\jobname.graphics%
\let\oincludegraphics\includegraphics% store original \includegraphics
\renewcommand{\includegraphics}[2][]{% prepend to it (could also use xpatch, etc.)
  \immediate\write\graphics{#2}
  \oincludegraphics[#1]{#2}}

\newcommand{\kommentti}[1]{
  {\bf[#1]}
}


\title{Automated Segmentation Exercises with PATKIT}
\author{Pertti Palo} 
\date{2,3,5 June 2025} 


\begin{document}

\frame{\titlepage
  \centering
} 


\frame{\frametitle{Land acknowledgement}

The University of Alberta, its buildings, labs and research stations are
primarily located on the territory of the Néhiyaw (Cree), Niitsitapi
(Blackfoot), Métis, Nakoda (Stoney), Dene, Haudenosaunee (Iroquois) and
Anishinaabe (Ojibway/Saulteaux), lands that are now known as part of Treaties
6, 7 and 8 and homeland of the Métis. The University of Alberta respects the
sovereignty, lands, histories, languages, knowledge systems and cultures of all
First Nations, Métis and Inuit nations. 
\vfill
I'd like to also add some words of my own.
}


\frame{\frametitle{Outline}
  \begin{itemize}
  \item Land acknowledgement
  \item This slide
  \item Introduction: The what and the why
  \item Method: The how
  \item Demo
  \item Want to have a go yourself?
  \item MaTiPSS ad
  \item Thanks and references
  \end{itemize}
}


\frame{\frametitle{Introduction: The what and the why}
  \begin{itemize}
  \item Segmentation can be explained, but actually getting good at it requires
  hands-on practice.
  \item While Praat \citep{BoersmaWeenink-PraatDoingPhonetics-2010} has an
  excellent segmentation interface, it does not provide a segmentation exercise
  interface.
  \item PATKIT \citep{PaloEtAl-PATKITPhoneticAnalysis-2025} copies Praat's
  segmentation interface and adds a resettable exercise feature. 
  \end{itemize}
}


\frame{\frametitle{Method: The how}
  \begin{itemize}
  \item  
  \end{itemize}
}


\frame{
  \centering
  {
    \bf \Large 
    \usebeamercolor[fg]{title}
    Demo - seeing is believing
    
    \vfill
  }
}


\frame{
  \centering
  {
    \bf \Large 
    \usebeamercolor[fg]{title}
    Want to have a go yourself?
    
    \vfill
  }
}


\frame{
  \centering
  {
    \bf \Large 
    \usebeamercolor[fg]{title}
    MaTiPSS ad
    
    \vfill
  }
}


\frame{
  \centering
  {
    \bf \Large 
    \usebeamercolor[fg]{title}
    Thank you!
    
    \vfill
  }
}


\frame{\frametitle{References}
  
\bibliographystyle{apalike}
\bibliography{main}

}

\end{document}

