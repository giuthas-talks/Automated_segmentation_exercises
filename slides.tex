\documentclass[12pt,a4,english,finnish,pdflatex%,handout
]{beamer}
\definecolor{MyGreen}{RGB}{50, 120, 50}
\usecolortheme[named=MyGreen]{structure}

\usepackage{babel}
\usepackage[utf8]{inputenc}
\usepackage[T1]{fontenc}
\usepackage{amsmath,amssymb} 
\usepackage{animate}
\usepackage{multimedia}

\usepackage{natbib}
\bibpunct[: ]{(}{)}{,}{}{}{;}

\usepackage{tikz}

\usepackage{tipa}

\usepackage{hyperref}

\setbeamertemplate{navigation symbols}{}

\graphicspath{{figures/}}

\setlength{\leftmargini}{0pt}
\setlength{\leftmarginii}{1em}

%% Write out the names of graphics files being included 
\newwrite\graphics
\immediate\openout\graphics=\jobname.graphics%
\let\oincludegraphics\includegraphics% store original \includegraphics
\renewcommand{\includegraphics}[2][]{% prepend to it (could also use xpatch, etc.)
  \immediate\write\graphics{#2}
  \oincludegraphics[#1]{#2}}

\newcommand{\kommentti}[1]{
  {\bf[#1]}
}


\title{Automated Segmentation Exercises with PATKIT}
\author{Pertti Palo} 
\date{3nd June 2025} 

\begin{document}

\frame{\titlepage
  \centering
} 


\frame{\frametitle{Land acknowledgement}

  The University of Alberta, its buildings, labs and research stations are
  primarily located on the territory of the Néhiyaw (Cree), Niitsitapi
  (Blackfoot), Métis, Nakoda (Stoney), Dene, Haudenosaunee (Iroquois) and
  Anishinaabe (Ojibway/Saulteaux), lands that are now known as part of Treaties
  6, 7 and 8 and homeland of the Métis. The University of Alberta respects the
  sovereignty, lands, histories, languages, knowledge systems and cultures of all
  First Nations, Métis and Inuit nations. 
  \vfill
  In addition to our university's written land acknowledgement I'd like to speak
  of my own relation to the lands where I have been working for almost a year now.

}


\frame{\frametitle{Outline}
  \begin{itemize}
  \item Land acknowledgement
  \item This slide
  \item Introduction: The what and the why
  \item The Caveats
  \item Method
  \item Demo - ish
  \item Discussion
  \item Want to have a go yourself?
  \item MaTiPS ad
  \item Thanks and references
  \end{itemize}
}


\frame{\frametitle{Introduction: The what and the why}
  \begin{itemize}
  \item Segmentation can be explained, but actually getting good at it requires
  hands-on practice.
  \item While Praat \citep{BoersmaWeenink-PraatDoingPhonetics-2010} has an
  excellent segmentation interface, it does not provide a segmentation exercise
  interface.
  \item The Phonetic Analysis ToolKIT (PATKIT, the program previously known as
  SATKIT) \citep{PaloEtAl-PATKITPhoneticAnalysis-2025} copies Praat's
  segmentation interface and adds a resettable exercise feature. 
  \end{itemize}
}


\frame{\frametitle{The Caveats}
  \begin{itemize}
    \item Exercises do not work yet.
    \item Everything might change.
    \item I want your input for making this good.
  \end{itemize}
}


\frame{\frametitle{Method: The workflow}
  \begin{itemize}
    \item Segment a bunch of data somewhere - for now probably in Praat.
    \item Alternatively, get a bunch of segmented data (TextGrids and e.g.
    wavs) from somewhere.
    \item Open the data directory in PATKIT as a new assignment.
    \item PATKIT scrambles the TextGrids.
    \item Segment away.
    \item Optional saving of answers to continue later and for things like
    inter-rater metrics. 
  \end{itemize}
}


\frame{\frametitle{Method: Installing PATKIT}
\begin{itemize}
  \item Install uv \url{https://docs.astral.sh/uv/}
  % \begin{itemize}
  %   \item Mac/Linux: 
  %   \begin{verbatim}
  %     curl -LsSf https://astral.sh/uv/install.sh | sh
  %   \end{verbatim} 
  %   \item Windows:
  %   \begin{verbatim}
  %     powershell -ExecutionPolicy ByPass -c "irm https://astral.sh/uv/install.ps1 | iex"
  %   \end{verbatim}
  % \end{itemize} 
  \item On the commandline run:\\
    uv tool install patkit\\
    patkit
  \item This should print the commandline help.
\end{itemize}
}


\frame{\frametitle{Method: Running an example assignment/exercise}
\begin{itemize}
  \item Get the example data from\\ 
  \url{https://github.com/giuthas-talks/Automated_segmentation_exercises/}
  \item Put it in a couple of folders of your own choosing. 
  \item Run:\\
    patkit exercise [folder name]
  \item This should open the annotator GUI.
\end{itemize}
}


\frame{\frametitle{Method: Potential command names}

\begin{itemize}
  \item Exercises $\rightarrow$ Create assignment\ldots
  \item Exercises $\rightarrow$ Save assignment\ldots 
  \item Exercises $\rightarrow$ New answer 
  \item Exercises $\rightarrow$ Open answer\ldots 
  \item Exercises $\rightarrow$ Save answer\ldots 
  \item Exercises $\rightarrow$ Compare to model
  \item Exercises $\rightarrow$ Show model
\end{itemize}
}


\frame{
  \centering
  {
    \bf \Large 
    \usebeamercolor[fg]{title}
    Demo
    \vfill
  }
}


\frame{\frametitle{Discussion}
\begin{itemize}
  \item So, do you have thoughts?
  \item What would make this more useful to you?
  \item What other features would be good to have?
\end{itemize}
}


\frame{
  \centering
  {
    \bf \Large 
    \usebeamercolor[fg]{title}
    Want to have a go yourself?
    \vfill
    --
    \vfill
    Keep an eye out for version 0.18 and updates (0.18.x) later this week.
    \vfill
  }
}


\frame{
  \centering
  {
    \bf \Large 
    \usebeamercolor[fg]{title}
    MaTiPS ad
    \vfill
  }
}


\frame{\frametitle{Methods and Techniques in Phonetic Sciences: MaTiPS}
  \begin{itemize}
    \item \url{https://matips.org/}
    \item Title and authors 19 July, anywhere on Earth
    \item Full one-page abstract 26 July, anywhere on Earth 
    \item Results by beginning of August.
    \item Conference 17-19 October, University of Alberta, Edmonton, Alberta, Canada.
  \end{itemize}
\vspace{0.5cm}
\begin{itemize}
  \item Canadian Acoustics Week happens to be in neighbouring Calgary 15-17 October.
  \begin{itemize}
    \item \url{https://jcaa.caa-aca.ca/index.php/jcaa/announcement/view/91}
    \item There will probably be a bus between the conferences.
  \end{itemize}
\end{itemize}
}


\frame{
  \centering
  {
    \bf \Large 
    \usebeamercolor[fg]{title}
    Thank you!
    
    \vfill
  }
}


\frame{\frametitle{References}
  
\bibliographystyle{apalike}
\bibliography{main}

}

\end{document}

